\documentclass{beamer}
\usepackage[spanish]{babel}
\usepackage[utf8]{inputenc} 

%\Titulo de la presentacion
\title{Objetivos:Generales y Particulares   \linebreak Formulación de Hipotesis}
%\Autores
\author{Ana Laura Lopez Flores 15211311 \linebreak Herrera Duran Osmar Said 1521307 \linebreak Beltran Medrano Hector Alonso 15211885 \linebreak Gutierrez Cruz Jorman Eliud 15211303 \linebreak Alvarado Rocha Ian Sebastian 15211265 \linebreak Gallardo Pérez Carlos David 15211292
	}
%\date[13-03]{Marzo-2016}
\date{18 de abril de 2016}

\usetheme{EastLansing}
\setbeamerfont{title}{shape=\itshape,family=\rmfamily}
\setbeamercolor{title}{fg=red!80!black,bg=blue!20!white}

\begin{document}
	\begin{frame}
		\titlepage
		\scriptsize
		\begin{center}
			Ingeniería en Sistemas Computacionales \\
			Instituto Tecnológico de Tijuana \\
		\end{center}
	\end{frame}
	
	\begin{frame}
		\frametitle{Contenido}
		%\tableofcontents[pausesections]
		\tableofcontents
	\end{frame}
	
	\begin{frame}
		\frametitle{}
		\begin{center}
		{\LARGE Objetivos generales y particulares	}
		\end{center}						
	\end{frame}
	
	\begin{frame}
		\frametitle{¿Qué es un objetivo?}
		\begin{definition}
			Un aspecto definitivo en todo proceso de investigación es la definición de los objetivos o del rumbo que debe tomar la investigación que va a realizarse. Así, los objetivos son los propósitos del estudio, expresan el fin que pretende alcanzarse; por tanto, todo el desarrollo del trabajo de investigación se orientara a lograr estos objetivos. (1)
			\linebreak 
		\end{definition}
		\begin{example}
			\begin{itemize}
				\item Objetivos Generales
				\item Objetivos Particulares 
			\end{itemize}
		\end{example}
	\end{frame}
		\begin{frame}
			\frametitle{¿Cuál es la manera de redactar y definir los objetivos?}
			\begin{definition}
			Puesto que todo objetivo implica la acción que se desea lograr, es importante tener en cuenta que al redactar los objetivos de la investigación deben utilizarse verbos en infinitivo. (2)	
			No es necesario escribir preámbulos al momento de redactar los objetivos; es recomendable expresar directamente el objetivo. Por ejemplo, si un objetivo es “hacer un análisis de la situación actual del sector de las artes gráficas en la ciudad de…, no es necesario agregar frases previas al objetivo como; “debido a que las empresas del sector de las artes gráficas atraviesan una situación económica difícil, en este estudio se pretende hacer un análisis…”. (2)
			\end{definition}
		\end{frame}
			\section{Objetivos generales y particulares}
			\subsection{Objetivos generales}
				\subsection{Objetivos Particulares}
			\begin{frame}
				\frametitle{¿Cuál es la manera de redactar y definir los objetivos?}
				\begin{definition}
					Otro aspecto muy importante en el momento de plantear los objetivos de la investigación es utilizar verbos que puedan lograrse o alcanzarse durante el desarrollo de esta (2), por lo cual, al redactarlos, es habitual utilizar verbos y derivados del tipo infinitivo.
				\end{definition}
			\end{frame}
			\section{Formulacion de hipotesis}
			
			\begin{frame}
				\frametitle{¿Cuál es la manera de redactar y definir los objetivos?}
				\begin{definition}
					El uso de verbos como capacitar, cambiar, motivar, ensenar, mejorar y muchos otros que implican acciones finales debe ser prudente, porque estas acciones casi nunca se logran durante el progreso de la investigación, debido a que implican dedicarles tiempo, recursos y, muchas veces, tomar decisiones para desarrollar el objetivo propuesto. (2)
					Respecto de los conceptos o variables incluidas. Evidentemente, los objetivos que se especifiquen deben ser congruentes entre sí. (1)					
				\end{definition}
			\end{frame}
			
	\begin{frame}
	 \begin{definition}
	 Definición: Se desprenden del general y deben formularse de forma que estén orientados al logro del objetivo general, es decir, que cada objetivo específico este diseñado para lograr un aspecto de aquel; y todos en su conjunto, la totalidad del objetivo general. Los objetivos específicos son los pasos que se dan para lograr el objetivo general (2). 
	 Estos son proposiciones que expresan con bastante claridad qué es lo que se va a hacer con los resultados de la investigación, también, pueden estar referidos a la obtención de resultados o a la realización de operaciones. (3) Son más puntuales y concretos, y en general indican conocimientos de menor complejidad, que se irán obteniendo durante la investigación, y que contribuirán a lograr el objetivo general. (4)
	\end{definition}
\end{frame}

\begin{frame}
	\frametitle{}
	\begin{center}
		{\LARGE Formulación de hipótesis }
	\end{center}						
\end{frame}

	 	
	\begin{frame}
		\frametitle{¿Qué es la formulación de hipótesis?}
		\begin{definition}
			Las hipótesis indican lo que estamos buscando o tratando de probar y pueden definirse como explicaciones tentativas del fenómeno investigado formuladas a manera de proposiciones (realización de una propuesta). 
			
			Las hipótesis no necesariamente    son Verladera, pueden o no serlo, pueden o no comprobarse con hechos. Son explicaciones tentativas, no los hechos en si. Al formularlas, el investigador no puede asegurar que vayan a comprobarse. Como mencionan y ejemplifican Black y Champion (1976), una hipótesis es diferente de una afirmación de un hecho.{1}
			
		\end{definition}
	
	\end{frame}
	\begin{frame}
		\frametitle{¿Qué es hipótesis?}
		\begin{definition}
		Suposición hecha a partir de datos que nos sirven como base para iniciar una investigación o una argumentación.
		\end{definition}
	\end{frame}
		\begin{frame}
			\frametitle{¿En toda investigación cuantitativa debemos
				plantear hipótesis?}
			\begin{definition}
				No, no en todas las investigaciones cuantitativas se plantean hipótesis. El hecho de que formulemos
				o no hipótesis depende de un factor esencial: el alcance inicial del estudio. Las investigaciones cuantitativas
				que formulan hipótesis son aquellas cuyo planteamiento define que su alcance será correlacional
				o explicativo, o las que tienen un alcance descriptivo, pero que intentan pronosticar una cifra
				o un hecho.{1}
			\end{definition}
			
		\end{frame}
	\begin{frame}
		\frametitle{¿Que son las variables?}
		\begin{definition}
			Una variable es una característica de algo susceptible de tomar más de un valor o de ser expresada en varias categorías. Las variables más comunes tomadas en la investigación social son la edad, sexo, la filiación étnica, la educación, los ingresos, el estatus matrimonial y la ocupación.{1}
			
		\end{definition}
	\end{frame}
		\begin{frame}
			\frametitle{¿De donde surgen las hipótesis?}
			\begin{definition}
			Existe pues, una relación muy estrecha entre el planteamiento del problema, la revisión de la literatura y las hipótesis. La revisión inicial de la literatura hecha para familiarizarnos con el problema de estudio nos lleva a plantearlo, después revisamos la literatura y afinamos o precisamos el planteamiento, del que derivan las  hipótesis. 
			
			“Al formular las hipótesis volvemos a evaluar nuestro planteamiento del   problema.”
			\end{definition}
		\end{frame}
		\begin{frame}
			\frametitle{Tipos de Hipótesis}
			\begin{definition}
				-Hipótesis de investigación (Hi)
				\linebreak-Hipótesis nulas (Ho)
				\linebreak-Hipótesis alternativas (Ha)
				\linebreak-Hipótesis estadísticas(He)	
			\end{definition}
		
		\end{frame}
			\begin{frame}
				\frametitle{Las hipótesis de investigación (Hi) pueden ser:}
				\begin{definition}
				  Las hipótesis de investigación, que se definen como proposiciones tentativas acerca de
					las posibles relaciones entre dos o más variables (Babbie, 2014 y 2012; Martin y
					Bridgmon, 2012; Davis, 2008; Kalaian y Kasim, 2008 e Iversen, 2003).{2}
					\linebreak-Descriptivas
					\linebreak-Correlacionales
					\linebreak-De diferencia de grupos
					\linebreak-De relación de causalidad
					
				\end{definition}
			
			\end{frame}
				\begin{frame}
					\frametitle{Hipótesis Descriptivas}
					\begin{definition}
						Describen el valor de las variables que se van a observar en un contexto.
						
					\end{definition}
					\begin{example}
						Hi: “La población de mangle rojo en Los Haitises oscila entre 40 y 50 por ciento”
						
					\end{example}
				\end{frame}
					\begin{frame}
						\frametitle{Hipótesis correlacionales}
						\begin{definition}
							Especifican relaciones entre dos o más variables.
							Importante: el orden en que se coloquen las variables no es importante
							
						\end{definition}
						\begin{example}
							Hi: “Los árboles de pino de mayor altura serán los de mayor grosor de tallo”
							
						\end{example}
					\end{frame}
						\begin{frame}
							\frametitle{Hipótesis de diferencia entre grupos}
							\begin{definition}
								Se formulan en investigaciones dirigidas a comparar dos o más grupos.
								
							\end{definition}
							\begin{example}
								Hi: “La calidad del agua vendida a granel es inferior a la del agua vendida embotellada”
								
							\end{example}
						\end{frame}
				  	    \begin{frame}
								\frametitle{Hipótesis de relaciones de causalidad}
								\begin{definition}
									No solamente proponen relación entre dos o más variables, sino que establecen el “sentido de entendimiento” entre ellas.
									
								\end{definition}
								\begin{example}
									Hi: “El nivel de educación y el nivel de ingresos influye sobre la disposición para pagar por la preservación de la calidad del agua del río Birán”
									
								\end{example}
							\end{frame}
							\begin{frame}
								\frametitle{Hipótesis Nulas (Ho)}
								\begin{definition}
									Si la Hi afirma que hay relación entre dos variables o dos grupos, la Ho niega esta relación.
									
								\end{definition}
								
							\end{frame}
							\begin{frame}
								\frametitle{Inducción y deducción}
								\begin{definition}
									La inducción se refiere al movimiento del pensamiento que va de los hechos particulares a afirmaciones de carácter general. 
									
									- Busca una comprensión más profunda en síntesis racionales (hipótesis, leyes, teorías). 
									
									- resulta de reunir distintos elementos que estaban dispersos o separados organizándolos y relacionándolos
									
									La deducción es el método que permite pasar de afirmaciones de carácter general a hechos particulares. 
									
									- Describir consecuencias desconocidas, de principios conocidos.
									
									- Encontrar principios desconocidos, a partir de otros conocidos. Una ley o principio puede reducirse a otra más general que la incluya.{1}
									
									
								\end{definition}
								
							\end{frame}
							\begin{frame}
								\frametitle{Hipótesis Alternativa (Ha)}
								\begin{definition}
									Son posibilidades “alternativas” ante las hipótesis de investigación y nula.
									Solo se formulan cuando efectivamente hay otras posibilidades adicionales a las hipótesis de investigación o nula.{2}
									
								\end{definition}
								
							\end{frame}
							\begin{frame}
								\frametitle{Hipótesis Estadísticas}
								\begin{definition}
									Prácticamente son las mismas hipótesis de investigación expresadas en forma estadística.
									
									Es decir, son transformadas en símbolos estadísticos y se pueden realizar cuando los datos a estudiar son mensurables (Se puede medir).{2}
									
									
								\end{definition}
							
							\end{frame}
							\begin{frame}
								\frametitle{Hipótesis Estadísticas División}
								\begin{definition}
								De estimación: 
								estas suponen el valor de alguna característica de la muestra que fue seleccionada y de la población en su conjunto. Para formularlas se tienen en cuenta datos adquiridos previamente.
								
								Estadísticas de correlación: 
								buscan establecer estadísticamente las relaciones existentes entre dos o más variables.	
								
								Estadísticas de la Diferencia de Medias u otros Valores:
								En este tipo de hipótesis se compara una estadística entre dos o mas grupos.{2}
								
							
								\end{definition}
								
							\end{frame}
								\begin{frame}
									\frametitle{Bibliografia:}
									\begin{definition}
									1- Formulación de hipótesis, recuperado de: https://sites.google.com/site/wikinfermeria/trabajo-final-de-grado/2-recursos-para-formulacion-de-hipotesis
									
									2- Metodologia-de-la-Investigacion-3edi-Bernal
									http://www.eumed.net/libros-gratis/2006c/203/1t.htm
									
										
									\end{definition}
									
								\end{frame}					
\end{document}